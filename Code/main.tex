\documentclass[10pt]{beamer}

\usepackage{amsmath}
\usepackage[utf8]{inputenc}  
\usepackage[T1]{fontenc}
\usetheme{Warsaw}

\usepackage{graphicx}
\usepackage{textcomp} 
\hypersetup{breaklinks=true}
\title[Stockage d'énergie]{Stocker de manière écologique, quantitative et durable de l’énergie}

\author{Pierre Minier}
\date{2020}

\begin{document}
	\begin{frame}[plain]
	\titlepage
	\end{frame}




	\section{Introduction}
	
	\begin{frame}[plain]
		\frametitle{Besoin de stocker de l'énergie}
		Objectif français : 32\% de renouvelable en 2030 (loi LTECV)
		\begin{figure}
			\includegraphics[scale = 0.35]{eolienne.jpg}
			\caption{Parc d'éolienne offshore}
		\end{figure}
Risques : pénuries et instabilité sur le réseau
	\end{frame}
    
    \begin{frame}[plain]{Plan}
        \begin{enumerate}
            \item Présentation d'un système de stockage
            \item Modélisation du système
            \item Rendement, Puissance et temps de décharge
            \item Expériences sur une maquette
            \item Application à une échelle réelle
        \end{enumerate}
    \end{frame}

    \section{Présentation d'un système de stockage}
    \subsection{Les stockages mécaniques}
	\begin{frame}
		\frametitle{Présentation d'un système de stockage}
		\framesubtitle{Les stockages mécaniques}
	\begin{columns}
		
		\begin{column}{5.5cm}
			\begin{figure}
				\includegraphics[scale=0.18]{STEP.png}
				\caption{STEP}
			\end{figure}
		\end{column}
		
		\begin{column}{6cm}
			\begin{itemize}
				\item Utilisation d'une énergie potentielle
				\newline
				\item Premier moyen de stockage stationnaire mondial
			\end{itemize}
		\end{column}
	\end{columns}
	\end{frame}

    \subsection{Lien avec l'océan}
	\begin{frame}[plain]
		\frametitle{Présentation d'un système de stockage}
		\framesubtitle{Lien avec l'océan}
		\begin{itemize}
			\item Stocker en hauteur $\Rightarrow$ stocker en profondeur
			\item \text{Utilisation de bouées cylindriques}
		\end{itemize}
	\[\]
		\begin{figure}
			\includegraphics[scale=1]{principe.png}
			\caption{schéma de principe}
		\end{figure}
	\end{frame}

	\subsection{Présentation d'une maquette}
	\begin{frame}[plain]
	\frametitle{Présentation d'un système de stockage}
	\framesubtitle{Présentation d'une maquette}
		\begin{figure}
			\includegraphics[scale = 0.72]{Modele.png}
			\caption{schéma de la maquette construite}
		\end{figure}
	\end{frame}


	\begin{frame}[plain]
	\frametitle{Présentation d'un système de stockage}
	\framesubtitle{Présentation d'une maquette}
	\begin{columns}
	\begin{column}{5 cm}
        \begin{figure}
            \includegraphics[scale = 0.31]{2.jpeg}
            \caption{Vue de biais}
        \end{figure}
    \end{column}
    \begin{column}{5cm}
        \begin{figure}
            \includegraphics[scale = 0.15]{3.jpeg}
            \caption{Vue de face}
        \end{figure}
    \end{column}
    \end{columns}
	\end{frame}




	\section{Modélisation du système}
	\subsection{Bilan de forces}
	\begin{frame}
	\frametitle{Modélisation du système}
	\framesubtitle{Bilan de force}
		\begin{columns}
			\begin{column}{5.5cm}
				\begin{figure}
				\includegraphics[scale=0.4]{{BilanDeForce.png}}
				\caption{Les forces appliquées}
				\end{figure}
			\end{column}
		
			\begin{column}{7.5cm}
				\begin{enumerate}
					\item Le Poids : $\vec{P} = m.\vec{g}$
					\item La poussée d'Archimède : $\vec{\pi} = - \rho_{eau}.V.\vec{g}$
					\item Une force de frottement : $\vec{F}_f = - k.\vec{v_i}^\alpha$
					\item Une force de traction (descente) : $\vec{F}_{tr}$
					\item Une force de tension (remonté) : $\vec{F}_{Te}$
				\end{enumerate}
			\end{column}
		\end{columns}
	\end{frame}
	\subsection{La poussée d'Archimède}
	\begin{frame}[plain]
	\frametitle{La poussée d'Archimède}
	\frametitle{Modélisation du système}
	\framesubtitle{La poussée d'Archimède}
	 	
	 	\begin{columns}
	 	\begin{column}{5.5cm}
	 	\begin{figure}
		 	\includegraphics[scale=0.55]{VolumeImmerge.png}
	 		\caption{Le volume immergé}
	 	\end{figure}
 		\end{column}
 	
 		\begin{column}{8.5 cm}
 			\begin{center}
 			$\boxed{\vec{\pi} = - \rho_{eau}.V(x).\vec{g}}$
 			\end{center}
 			\[\]
 			$
 			V(x) = \left\{
 			\begin{array}{r c l}
 			\pi.r_{ba}^2.h_{ba}&&0\leq x \leq n-h_{ba}\\
 			\pi.r_{ba}^2.(n-x)&&n-h_{ba}\leq x \leq n
 			\end{array}
 			\right.
 			$
 			
 		\end{column}
 	\end{columns}
	\end{frame}

	\subsection{La force de frottement}
	\begin{frame}[plain]
		\frametitle{Modélisation du système}
		\framesubtitle{La force de frottement}
		\begin{center}
			 $\boxed{Re = \frac{||\vec{v}_{ba}||.r_{ba}.\rho_{eau}}{\eta_{eau}}}$
		\end{center}
	\[\]
		\begin{figure}
			\includegraphics[scale=0.7]{RegimeReynolds.png}
			\caption{Modèles de frottement selon le nombre de Reynolds}
		\end{figure}
	
	\end{frame}

	\begin{frame}[plain]
	\frametitle{Modélisation du système}
	\framesubtitle{La force de frottement}
		\begin{figure}
			$\boxed{k = \frac{C_D}{2}.\rho_{eau}.||\vec{v}_{ba}||^2.\pi.r_{ba}^2}$
		\[\]
		\includegraphics[scale=1]{abaque.png}
		\caption{Coefficient de traîné ($C_D$)}
		\end{figure}
	\end{frame}





	\section{Rendement, Puissance et temps de décharge}
		\subsection{Caractériser une méthode de stockage}
	\begin{frame}
	\frametitle{Rendement, Puissance et temps de décharge}
	\framesubtitle{Caractériser une méthode de stockage}
		\begin{columns}
			\begin{column}{7cm}
				\begin{figure}
					\includegraphics[scale=0.1]{Conclu1.png}
					\caption{Classification des moyens de stockage}
				\end{figure}
			\end{column}
			
			\begin{column}{5cm}
				\begin{itemize}
					\item Le rendement énergétique
					\item Le temps de décharge/charge
					\item La puissance récupérée
				\end{itemize}
			\end{column}
			\end{columns}
	\end{frame}


	\subsection{A vitesse constante}
	\begin{frame}[plain]
	\frametitle{Rendement énergétique}
	\framesubtitle{A vitesse constante : la descente}
		\begin{columns}
			\begin{column}{4cm}
				\includegraphics[scale=0.6]{BDFDescente.png}
			\end{column}
		
			\begin{column}{6cm}
				\[
				\begin{array}{r c l}
				PFD \Rightarrow ||\vec{F}_{tr}|| &=& ||\vec{\pi}|| + ||\vec{F}_f|| - ||\vec{P}||\\\\
				E_{consomm \acute{e} e} &\stackrel{d \acute {e}f}=& \int_{0}^{t_f}\vec{F}_{tr}.\vec{v}_ddt
				\end{array}
				\]
				\[\]
				$\boxed{E_{consomm \acute{e} e} = ||\vec{v}_d||.\int_{0}^{t_f}||\vec{F}_{tr}||dt}$

			\end{column}
		\end{columns}
	\end{frame}



	\begin{frame}[plain]
	\frametitle{Rendement énergétique}
	\framesubtitle{A vitesse constante : la remontée}
	\begin{columns}
		\begin{column}{4cm}
			\includegraphics[scale=0.6]{BDFRemonte.png}
		\end{column}
		
		\begin{column}{6cm}
			\[
			\begin{array}{r c l}
			PFD \Rightarrow ||\vec{F}_{te}|| &=& ||\vec{\pi}|| - ||\vec{F}_f|| - ||\vec{P}||\\\\
			E_{r \acute{e} cup \acute{e} r \acute{e} e} &\stackrel{d \acute{e} f}=& \int_{0}^{t_f}\vec{F}_{te}.\vec{v}_rdt
			\end{array}
			\]
			\[\]
			$\boxed{E_{r \acute{e} cup \acute{e} r \acute{e} e} = -||\vec{v}_r||.\int_{0}^{t_f}||\vec{F}_{te}||dt}$
		
	\end{column}
	\end{columns}
	\end{frame}
	\begin{frame}[plain]
	\frametitle{Rendement énergétique}
	\framesubtitle{A vitesse constante : évaluation du rendement}
		\begin{figure}
			\includegraphics[scale=0.38]{17.png}
			\caption{carte de fréquentation}
		\end{figure}
	\end{frame}
	
	\subsection{Étude du régime libre}
	\begin{frame}
	\frametitle{Rendement, Puissance et temps de décharge}
	\framesubtitle{Étude du régime libre}
		\begin{center}
			$PFD \Rightarrow\boxed{m_{ba}.\ddot{x}_{ba} = \rho_{eau}.V(x_{ba}).g - m_{ba}.g - k(\dot{x}_{ba}).\dot{x}_{ba}^\alpha}$
		\end{center}	
		
		
		\[\]Plan d'étude :
		\begin{enumerate}
			\item Condition imposée par un régime libre - permanent
			\item Résolution "à la main" dans un cas simplifié
			\item Résolution par une méthode numérique
		\end{enumerate}
	\end{frame}

	
	\begin{frame}[plain]
	\frametitle{Étude du régime libre}
	\framesubtitle{Condition imposée par un régime libre - permanent}

		\begin{center}
            $\boxed{k(v_{\infty}).v_{\infty}^\alpha = g.(\rho_{eau}.\pi.r_{ba}^2.h_{ba} - m_{ba})} \quad \Rightarrow \textit{Densité du ballast} < 1$
		\end{center}
		
			\begin{figure}
			\includegraphics[scale=0.8]{5.png}
			\caption{Vitesse "infinie" selon la masse}
			\end{figure}
	\end{frame}

	

	\begin{frame}[plain]
	\frametitle{Étude du régime libre}
	\framesubtitle{Résolution de l'équation différentielle dans un cas simplifié}

	\begin{center}
	$\boxed{m_{ba}.\ddot{x}_{ba} + k.\dot{x}_{ba}^\alpha = (\rho_{eau}.\pi.r_{ba}^2.h_{ba} - m_{ba}).g} \quad \alpha \neq 1$
	\end{center}
	\quad\quad\quad\quad\quad\quad Linéarisation en posant : $\dot{x}_{ba} = Y^{\frac{1}{1 - \alpha}}$
	
	\begin{figure}
		\includegraphics[scale=0.72]{6.png}
		\caption{Solution théorique avec k constant}
	\end{figure}
	
	\end{frame}


	\begin{frame}[plain]
	\frametitle{Étude du régime libre}
	\framesubtitle{Résolution par une méthode numérique}
		\begin{center}
		$\boxed{m_{ba}.\ddot{x}_{ba} + k(\dot{x}_{ba}).\dot{x}_{ba}^\alpha - \rho_{eau}.V(x_{ba}).g = - m_{ba}.g}$
		\end{center}
		\begin{figure}
		\includegraphics[scale=0.7]{12.png}
		\caption{Vitesse de remontée en régime libre}
		\end{figure}
	\end{frame}
    
    \subsection{Conclusion de la partie théorique}
    \begin{frame}
    \frametitle{Rendement, Puissance et temps de décharge}
    \framesubtitle{Conclusion de la partie théorique}
            Modèle de la vitesse:
            \begin{enumerate}
                \item Vitesse "créneau"
                \item Vitesse du régime permanent
            \end{enumerate}
            \[\]
        	Conséquences :
        	\begin{enumerate}
        	    \item Accès au rendement énergétique
        	    \item Formules simples valides:
            \begin{center}
                \begin{itemize}
                      \item ${T_{d \acute{e} charge}} = \frac{Profondeur}{v_{ba}}$
                      \item $P_{d \acute{e} livr \acute{e} e} = \frac{E_{r \acute{e} cup \acute{e} r \acute{e} e }}{T_{d \acute{e} charge}}$
                    \end{itemize}
                  \end{center}
        	\end{enumerate}

        
    \end{frame}

	\section{Expériences sur une maquette}
	\subsection{Logiciel utilisé}
    \begin{frame}
    \frametitle{Expérience sur la maquette}
    \framesubtitle{Logiciel utilisé}
    \begin{figure}
        \centering
        \includegraphics[scale = 0.25]{tracker.png}
        \caption{Utilisation du logiciel tracker}
    \end{figure}
        
    \end{frame}
    
    \subsection{Expérience 1}
	\begin{frame}[plain]
	\frametitle{Expérience sur la maquette}
	\framesubtitle{Utilisation d'une vitesse angulaire}
	\begin{figure}
	    \centering
	    \includegraphics[scale = 0.7]{16.png}
	    \caption{Résultat de l'expérience 1}
	\end{figure}
	    
	\end{frame}
	
	\subsection{Expérience 2}
	\begin{frame}[plain]
	\frametitle{Expérience sur la maquette}
	\framesubtitle{Expérience 2 : sans la structure annexe}
		\begin{figure}
		    \centering
		    \includegraphics[scale = 1]{Experience2.png}
		    \caption{Modification de la maquette}
		\end{figure}
	\end{frame}
	
	\begin{frame}[plain]
	\frametitle{Expérience sur la maquette}
	\framesubtitle{Expérience 2 : sans la structure annexe}
	\begin{figure}
	    \centering
	    \includegraphics[scale = 0.65]{14.png}
	    \caption{Résultat de l'expérience 2}
	\end{figure}
	\end{frame}
	
	\subsection{Expérience 3}
	\begin{frame}[plain]
	\frametitle{Expérience sur la maquette}
	\framesubtitle{Expérience 3 : s'affranchir des effets de bords}
	\begin{columns}
	    \begin{column}{6cm}
        	\begin{figure}
        	    \centering
        	    \includegraphics[scale = 0.05]{piscine.jpg}
        	    \caption{Expérience dans une piscine}
        	\end{figure}
        \end{column}
        \begin{column}{6cm}
        	\begin{figure}
        	    \centering
        	    \includegraphics[scale = 0.31]{ballast.png}
        	    \caption{Le ballast muni de "l'antenne"}
        	\end{figure}        
        \end{column}
    \end{columns}
	\end{frame}	
	
	
	\begin{frame}[plain]
	\frametitle{Expérience sur la maquette}
	\framesubtitle{Expérience 3 : s'affranchir des effets de bords}
	\begin{figure}
	    \centering
	    \includegraphics[scale = 0.65]{15.png}
	    \caption{Résultat de l'expérience 3}
	\end{figure}
	\end{frame}
	
    \begin{frame}[plain]
	\frametitle{Expérience sur la maquette}
	\framesubtitle{Expérience 3 : s'affranchir des effets de bords}
	\begin{figure}
	    \centering
	    \includegraphics[scale = 0.75]{Figure_3.png}
	    \caption{Ajout des incertitudes}
	\end{figure}
	\end{frame}


	\section{Application à une échelle réelle}
	\subsection{Démarche}
	\begin{frame}
	    \frametitle{Application à une échelle réelle}
	    \framesubtitle{Démarche effectuée}
	    
	    \begin{figure}
	        \centering
	        \includegraphics{Raisonnement.png}
	        \caption{Raisonnement}
	    \end{figure}
	\end{frame}
	\subsection{Puissance et temps de décharge}
	\begin{frame}[plain]
	\frametitle{Application}
	\framesubtitle{Puissance et temps de décharge}

	   
	   \begin{figure}
	       \centering
	       \includegraphics[scale = 0.33]{(5) Déchargementpuissance.png}
	       \caption{Évaluation des ordres de grandeurs}
	   \end{figure}
	\end{frame}
	
	\begin{frame}[plain]
	\frametitle{Application à une échelle réelle}
	\framesubtitle{Puissance et temps de décharge}
	    \begin{figure}
	        \centering
	        \includegraphics[scale = 0.145]{Conclu2.png}
	         \caption{Comparaison avec des technologies existantes}
	    \end{figure}
	\end{frame}
	
	\subsection{Conclusion}
	\begin{frame}
        \frametitle{Application à une échelle réelle}
        \framesubtitle{Conclusion}
            \begin{itemize}
                \item Puissance de 10 MW sur quelques minutes pour une centaine de mètre de profondeur
                \item Rendement de l'ordre de 40\%
                \item Utilisation proche des côtes, voire en pleine mer
            \end{itemize}
            \begin{figure}
            \centering
            \includegraphics[scale = 0.27]{floating-city-1024x598 - Copie.jpg}
            
            \end{figure}
	\end{frame}
	\begin{frame}
        \frametitle{Application à une échelle réelle}
        \framesubtitle{Conclusion}
            \begin{itemize}
                \item Puissance de 10 MW sur quelques minutes pour une centaine de mètre de profondeur
                \item Rendement de l'ordre de 40\%
                \item Utilisation proche des côtes, voire en pleine mer
            \end{itemize}
            \begin{figure}
            \centering
            \includegraphics[scale = 0.15]{floating-city-1024x598.jpg}
            \caption{Une ville flottante}
            \end{figure}
	\end{frame}
	

\end{document}
